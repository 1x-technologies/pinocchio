%%%%%%%%%%%%%%%%%%%%%%%%%%%%%%%%%%%%%%%%%%%%%%%%%%%%%%%
% Pinocchio Cheat Sheet
%
% Edited by Aurélie Bonnefoy
% Available at https://www.overleaf.com/project/5e66106889524e000114b5df
%
%%%%%%%%%%%%%%%%%%%%%%%%%%%%%%%%%%%%%%%%%%%%%%%%%%%%%%%

%%%%%%%%%%%%% TODOs %%%%%%%%%%%%%
% - purpose of the cheat sheet ? overview for beginners vs. completeness and adv. algos
% - add derivatives
% - add missing algorithms
% - restructure main algorithm section
% - think about common abbreviations to avoid too many 2liners
% - add links to overview / pinocchio doc / more detailed explanations, featherstone book
% - adapt design, maybe not all cols need to be the same lengths, algos need much more space
%%%%%%%%%%%%%%%%%%%%%%%%%%%%%%%%%

\documentclass[10.5pt]{article}
\usepackage[landscape,a3paper]{geometry}
\usepackage{url}
\usepackage{multicol}
\usepackage{amsmath}
\usepackage{amsfonts}
\usepackage{tikz}
\usetikzlibrary{decorations.pathmorphing}
\usepackage{amsmath,amssymb}

\usepackage{colortbl}
\usepackage{xcolor}
\usepackage{mathtools}
\usepackage{amsmath,amssymb}
\usepackage{enumitem}

\title{Pinocchio Cheat Sheet}
\usepackage[english]{babel}
\usepackage[utf8]{inputenc}

\usepackage{minted}
\usemintedstyle{vs}
\usepackage{xcolor}


\advance\topmargin-3cm
\advance\textheight3in
\advance\textwidth5in
\advance\oddsidemargin-6cm
%\advance\evensidemargin-1.5in
\parindent0pt
\parskip2pt
\newcommand{\hr}{\centerline{\rule{3.5in}{1pt}}}
%\colorbox[HTML]{e4e4e4}{\makebox[\textwidth-2\fboxsep][l]{texto}
\begin{document}

\begin{center}{\huge{\textbf{Pinocchio Cheat Sheet}}}
\end{center}
\begin{multicols*}{3}
\vspace{-1cm}

\tikzstyle{mybox} = [draw=black, fill=white, very thick,
    rectangle, rounded corners, inner sep=10pt, inner ysep=10pt]
\tikzstyle{fancytitle} =[fill=black, text=white, font=\bfseries]

%------------ Begin Get started ---------------------
\begin{tikzpicture}
\node [mybox] (box){%
    \begin{minipage}{0.9\linewidth}
        \begin{tabular}{p{3cm} | p{8cm}}
            import & \texttt{import pinocchio as pin}\\
            & \texttt{from pinocchio.utils import *} \\
            get documentation & \texttt{pin.Model?} \\
        \end{tabular}
    \end{minipage}
};

\node[fancytitle] at (box.north) {Get started};
\end{tikzpicture}
%------------ End Get started ---------------------

%------------ Begin Spatial quantities ---------------------
\begin{tikzpicture}
\node [mybox] (box){%
    \begin{minipage}{0.9\linewidth}

        \centerline{\textbf{Transforms}}\vspace{0.1cm}
        \begin{tabular}{p{4cm} | p{7cm}}
            SE3 & \texttt{aMb = pin.SE3(aRb,apb)} \\ \hline
            \quad unit transformation & \texttt{M = pin.SE3(1)} \\ 
            \quad rotation matrix & \texttt{M.rotation} \\
            \quad translation vector & \texttt{M.translation} \\ \hline
            SE3 inverse & \texttt{bMa = aMb.inverse()} \\
            SE3 action & \texttt{aMc = aMb * bMc} \\
            action matrix & \texttt{aXb = aMb.action} \\
            homegeneous matrix & \texttt{aHb = aMb.homogeneous} \\
        \end{tabular} \newline

        \centerline{\textbf{Spatial Velocity}}\vspace{0.1cm}
        \begin{tabular}{p{4cm} | p{7cm}}
            Motion & \texttt{m = pin.Motion(v,w)} \\ \hline
            \quad linear acceleration & \texttt{m.linear} \\
            \quad angular acceleration & \texttt{m.angular} \\ \hline
            SE3 action & \texttt{v\_a = aMb * v\_b} \\
        \end{tabular} \newline
        
        \centerline{\textbf{Spatial Acceleration}}\vspace{0.1cm}
        \begin{tabular}{p{4cm} | p{7cm}}
            is used in algorithms  & \texttt{a = ($\dot{\omega}$,$\dot{v}_O$)} \\ 
            Get classical acceleration & \texttt{a' = a + $(0, \omega \times v_O)$} \\
                          & \texttt{pin.classicAcceleration(v,a, [aMb])} \\
        \end{tabular} \newline

        \centerline{\textbf{Spatial Force}}\vspace{0.1cm}
        \begin{tabular}{p{4cm} | p{7cm}}
            Force & \texttt{f = pin.Force(l,n)} \\ \hline
            \quad linear force & \texttt{f.linear} \\
            \quad torque & \texttt{f.angular} \\ \hline
            SE3 action & \texttt{f\_a = aMb * f\_b} \\
        \end{tabular} \newline

        \centerline{\textbf{Spatial Inertia}}\vspace{0.1cm}
       \begin{tabular}{p{4cm} | p{7cm}}	
            Inertia & \texttt{Y = pin.Inertia(mass,com,I)} \\ \hline
            \quad mass & \texttt{Y.mass} \\
            \quad center of mass pos. & \texttt{Y.lever} \\
            \quad rotational inertia & \texttt{Y.inertia} \\
        \end{tabular} \newline
        
        \centerline{\textbf{Geometry}}\vspace{0.1cm}
        \begin{tabular}{p{4cm} | p{7cm}}
            Quaternion & \texttt{quat = pin.Quaternion(R)} \\
            Angle Axis & \texttt{aa = pin.AngleAxis(angle,axis)} \\ 
        \end{tabular} \newline
        
        \centerline{\textbf{Useful converters}} 
        \vspace{0.2cm}
        \begin{tabular}{p{4cm} | p{7cm}}
        	  SE3 $\rightarrow$ (x,y,z,quat) & \texttt{pin.se3ToXYZQUAT(M)} \\
        	  (x,y,z,quat) $\rightarrow$ SE3 & \texttt{pin.XYZQUATToSE3(vec)} \\
        \end{tabular}

    \end{minipage}
};

\node[fancytitle] at (box.north) {Spatial quantities};
\end{tikzpicture}
%------------ End Spatial quantities ---------------------

%------------ Begin Model --------------------------------
% Some items in this box here seems kind of self-explanatory, good idea to make award of the different attributes and methods but maybe not to best representation (model name - model.name) , others are quite useful.
\begin{tikzpicture}
\node [mybox] (box){%
    \begin{minipage}{0.9\linewidth}
        \begin{tabular}{p{4cm} | p{7cm}}
            Model & \texttt{model = pin.Model()} \\ \hline
            \quad model name & \texttt{model.name} \\
            \quad joint names & \texttt{model.names} \\
            \quad joint models & \texttt{model.joints} \\
            \quad joint placements & \texttt{model.placements} \\
            \quad link inertias & \texttt{model.inertias} \\
            \quad frames & \texttt{model.frames} \\ 
            Methods &  \\ \hline
            % Inputs missing, but there are a lot. maybe enough like this and doc should be ask. maybe another hint to the doc ?
            \quad add joint & \texttt{model.addJoint} \\
            \quad append body & \texttt{model.appendBodyToJoint} \\
            \quad add frame & \texttt{model.addFrame} \\
            \quad append child into parent model & \texttt{model.appendModel} \\
            \quad build reduced body & \texttt{model.buildReducedModel} \\
            % findCommonAncestor
        \end{tabular} 

    \end{minipage}
};

\node[fancytitle] at (box.north) {Model};
\end{tikzpicture}
%------------ End Model --------------------------------

%------------ Begin Parsers ----------------------------
\begin{tikzpicture}
\node [mybox] (box){%
    \begin{minipage}{0.9\linewidth}
        
         \begin{tabular}{p{3cm} | p{8cm}}
            load an URDF file & \texttt{pin.buildModelFromUrdf(filename,[root\_joint])} \\
            load a SDF file & \texttt{pin.buildModelFromSdf(filename,[root\_joint]} \\
             & \texttt{root\_link\_name,parent\_guidance)} \\
        \end{tabular} 
        
    \end{minipage}
};

\node[fancytitle] at (box.north) {Parsers};
\end{tikzpicture}
%------------ End Parsers --------------------------------

%------------ Begin Data ---------------------------------
\begin{tikzpicture}
\node [mybox] (box){%
    \begin{minipage}{0.9\linewidth}
        \begin{tabular}{p{4cm} | p{7cm}}
            Data & \texttt{data = pin.Data(model)} \\ 
             & \texttt{data = model.createData()} \\ \hline
            \hspace{0.5em} joint data & \texttt{data.joints} \\
            \hspace{0.5em} joint placements & \texttt{data.oMi} \\
            \hspace{0.5em} joint velocities & \texttt{data.v} \\
            \hspace{0.5em} joint accelerations & \texttt{data.a} \\
            \hspace{0.5em} joint forces & \texttt{data.f} \\
            \hspace{0.5em} mass matrix & \texttt{data.M} \\
            \hspace{0.5em} non linear effects & \texttt{data.nle} \\
            \hspace{0.5em} centroidal momentum & \texttt{data.hg} \\
            \hspace{0.5em} centroidal matrix & \texttt{data.Ag} \\
            \hspace{0.5em} centroidal inertia & \texttt{data.Ig} \\ \hline
            Reference frames &  \\
            LOCAL & local coordinate system of the joint \\
            LOCAL\_WORLD\_ALIGNED & local coordinate system aligned with WORLD axis \\
            WORLD & world coordinate system \\
        \end{tabular} 
        
    \end{minipage}
};

\node[fancytitle] at (box.north) {Data};
\end{tikzpicture}
%------------ End Data ---------------------------------


%------------ Begin Main Algorithms---------------------

% List the missing algorithms and decide which to include
% - joint : dIntegrate, interpolate, difference
% - contact-jacobian: getConstraintJacobian
% - contact-dynamics: forwardDynamics, impulseDynamics, computeKKTContactDynamicMatrixInverse
% - constrained-dynamics: initConstraintDynamics, constraintDynamics
% - constrained-dynamics-derivatives: computeConstraintDynamicsDerivatives
% - collision / broadphase: computeCollisions
% - kinematic regressor
% - regressor: computeStaticRegressor, bodyRegressor, jointBodyRegressor, computeJointTorqueRegressor
% - cholesky: decompose, solve, computeMinv
% - impulse dynamics

\begin{tikzpicture}
\node [mybox] (box){%
    \begin{minipage}{0.9\linewidth}

        \centerline{\textbf{Configuration}}
        \begin{tabular}{p{3cm} | p{8cm}}
            placement collision obj & \texttt{pin.updateGeometryPlacements(model, data, robot.collision\_model, robot.collision\_data, q)} \\ 
            collisions detection & \texttt{pin.computeCollisions(robot.collision\_model, robot.collision\_data, False)} \\
            distance from collision & \texttt{pin.computeDistances(robot.collision\_model, robot.collision\_data)} \\
            integrate configuration & \texttt{pin.integrate(model, q, v)} \\ 
        \end{tabular} \newline

        \centerline{\textbf{Kinematics}}
        \begin{tabular}{p{3cm} | p{8cm}}
            forward kinematics & \texttt{pin.forwardKinematics(model, data, q, [v, [ a] ])} \\
            forward kinematics derivatives & \texttt{pin.computeForwardKinematicsDerivatives( \newline model,data,q,v,a)} \\
            dv\_dq,dv\_dv = & \texttt{pin.getJointVelocityDerivatives(model,data, \newline joint\_id,pin.ReferenceFrame.WORLD)}   \\
            dv\_dq\_local,da\_dq\_\newline local,da\_dv\_local,\newline da\_da\_local = & \texttt{ pin.getJointAccelerationDerivatives(model,data,\newline joint\_id,pin.ReferenceFrame.LOCAL)} \\ 
            frames placement & \texttt{pin.framesForwardKinematics(model, data, q)} \\
            frame jacobian & \texttt{pin.computeFrameJacobian(model, data, q, frame\_id, ref\_frame)} \\ 
            % getVelocity, getAcceleration, getClassicalAcceleration
        \end{tabular} \newline

        \centerline{\textbf{Jacobian}}
        \begin{tabular}{p{3cm} | p{8cm}}
            full model Jacobian data.J & \texttt{pin.computeJointJacobians(model, data, [q])} \\
            joint Jacobian & \texttt{pin.getJointJacobian(model, data, joint\_id, ref\_frame)} \\
            full model dJ/dt & \texttt{pin.computeJointJacobiansTimeVariation(model, data, q, v)} \\
            joint dJ/dt & \texttt{pin.getJointJacobianTimeVariation(model, data, joint\_id, ref\_frame)} \\
        \end{tabular} \newline

        \centerline{\textbf{Forward Dynamics}}
        \begin{tabular}{p{3cm} | p{8cm}}mmmm
            Articulated-Body Algorithm $\ddot{q}$ & \texttt{pin.aba(model, data, q, v, tau, [f\_ext]) } \\
            Joint Space Inertia Matrix Inv & \texttt{pin.computeMinverse(model, data, [q]) } \\ \hline
            Composite Rigid-Body Algorithm & \texttt{pin.crba(model, data, q)}
        \end{tabular} \newline
        
        \centerline{\textbf{Inverse Dynamics}}
        \begin{tabular}{p{3cm} | p{8cm}}
            Recursive Newton-Euler Algorithm & \texttt{pin.rnea(model, data, q, v, a, [f\_ext]) } \\
            dtau\_dq, dtau\_dv, dtau\_da & \texttt{pin.computeRNEADerivatives(model, data, q, v, a, [f\_ext]) } \\
        % nonLinearEffects, computeGeneralizedGravity, computeStaticTorque, computeCoriolisMatrix
        \end{tabular} \newline
        
        \centerline{\textbf{Centroidal}}
        \begin{tabular}{p{3cm} | p{8cm}}
            Centroidal momentum & \texttt{pin.computeCentroidalMomentum(model, data, [q, v]) } \\
            Centroidal momentum + time derivatives & \texttt{pin.computeCentroidalMomentumTimeVariation(model, data, [q, v, a]) } \\
        % also mention ccrba, dccrba, computeCentroidalMap, computeCentroidalMapTimeVariation?
        \end{tabular} \newline
        
        \centerline{\textbf{Energy}}
        \begin{tabular}{p{3cm} | p{8cm}}
            forward kinematics and kin. E & \texttt{pin.computeKineticEnergy(model, data, [q, v]) } \\
            forward kinematics and pot. E & \texttt{pin.computePotentialEnergy(model, data, [q, v]) } \\
            forward kinematics and mech. E & \texttt{pin.computeMechanicalEnergy(model, data, [q, v]) } \\
        % also mention ccrba, dccrba, computeCentroidalMap, computeCentroidalMapTimeVariation?
        \end{tabular} \newline
        
        \centerline{\textbf{General}}
        \begin{tabular}{p{3cm} | p{8cm}}
            all terms, check doc & \texttt{pin.computeAllTerms(model, data, q, v) } \\
        \end{tabular} \newline


    \end{minipage}
};

\node[fancytitle, right=10pt] at (box.north west) {Main algorithms};
\end{tikzpicture}
%------------ End Main Algorithms------------------

%------------ Begin COM ---------------------------------
\begin{tikzpicture}
\node [mybox] (box){%
    \begin{minipage}{0.9\linewidth}
        \begin{tabular}{p{4cm} | p{7cm}}
            Total mass of model & \texttt{pin.computeTotalMass(model, [data])} \\ 
            Mass of each subtree & \texttt{pin.computeSubtreeMasses(model, data)} \\ 
            Center of mass & \texttt{pin.centerOfMass(model, data, q, [v, a],[compute\_subtree\_com])} \\ 
            Jacobian center of mass & \texttt{pin.jacobianCenterOfMass(model, data, [q],[compute\_subtree\_com])} \\ 
        \end{tabular} 
        
    \end{minipage}
};

\node[fancytitle] at (box.north) {COM};
\end{tikzpicture}
%------------ End COM ---------------------------------

%------------ Begin Frames ---------------------------------
\begin{tikzpicture}
\node [mybox] (box){%
    \begin{minipage}{0.9\linewidth}
        \begin{tabular}{p{4cm} | p{7cm}}
            placement of all operational frames & \texttt{pin.updateFramePlacements(model, data)} \\ 
            frame veloctiy & \texttt{pin.getFrameVelocity(model, data, frame\_id, reference\_frame)} \\ 
            frame acceleration & \texttt{pin.getFrameAcceleration(model, data, frame\_id, reference\_frame)} \\
            frame acceleration & \texttt{pin.getFrameClassicalAcceleration(model, data, frame\_id, reference\_frame)} \\
            frames placement & \texttt{pin.framesForwardKinematics(model, data, q)} \\
            frame jacobian & \texttt{pin.computeFrameJacobian(model, data, q, frame\_id, ref\_frame)} \\ 
            frame jacobian time variation & \texttt{pin.frameJacobianTimeVariation(model, data, q, v, frame\_id, ref\_frame)} \\ 
        \end{tabular} 
        
    \end{minipage}
};

\node[fancytitle] at (box.north) {Frames};
\end{tikzpicture}
%------------ End Frames ---------------------------------


%%------------ Optimization ---------------------
%\begin{tikzpicture}
%\node [mybox] (box){%
%    \begin{minipage}{0.9\linewidth}
%
%
%        \begin{tabular}{p{3cm} | p{8cm}}
%            initialization & \texttt{from scipy.optimize import fmin\_bfgs} \\
%            & \texttt{from scipy.optimize import fmin\_slsqp} \\ \hline
%            BFGS optimization & \texttt{fmin\_bfgs(cost, x0, callback)} \\ \hline
%            SLSQP optimization & \texttt{fmin\_slsqp(cost, x0, callback, f\_eqcons) }
%        \end{tabular}
%
%
%    \end{minipage}
%};
%%------------ Title ---------------------
%\node[fancytitle, right=10pt] at (box.north west) {Optimization};
%\end{tikzpicture}


%------------ Begin Robot Wrapper ---------------
% TODO explain usage / utility of robot wrapper in a sentence
\begin{tikzpicture}
\node [mybox] (box){%
    \begin{minipage}{0.9\linewidth}

        \begin{tabular}{p{3cm} | p{8cm}}
            import & \texttt{from pinocchio.robot\_wrapper import RobotWrapper} \\ \hline
            reference initial position & \texttt{robot.q0} \\ \hline
            display & \texttt{robot.display(q)} \\ \hline
            return joint index & \texttt{robot.index('joint name')} \\ \hline
            model constants & \texttt{model = robot.model} \\
            \quad names & \texttt{model.names} \\
            \quad frames & \texttt{model.frames} \\ \hline
            model data & \texttt{data = robot.data} \\ \hline
            joint placement & \texttt{robot.placement(idx)} \\
            \quad translation & \texttt{robot.placement(idx).translation} \\
            \quad rotation & \texttt{robot.placement(idx).rotation} \\ \hline
            frame placement & \texttt{robot.framePlacement(idx)} \\
        \end{tabular} \newline

    \end{minipage}
};

\node[fancytitle, right=10pt] at (box.north west) {Robot Wrapper class };
\end{tikzpicture}
%------------ End Robot Wrapper ---------------


%------------ Begin Gepetto Viewer --------------
\begin{tikzpicture}
\node [mybox] (box){%
    \begin{minipage}{0.9\linewidth}

        \centerline{\textbf{Get started}}
        \begin{tabular}{p{3cm} | p{8cm}}
            open from terminal & \texttt{gepetto-gui} \\ \hline
            import & \texttt{import gviewserver}\\ \hline
            initialization & \texttt{gv = gviewserver.GepettoViewerServer()} \\
            \quad or & \texttt{gv=robot.viewer.gui}
        \end{tabular} \newline

        \centerline{\textbf{Add basic shapes}}
        \begin{tabular}{p{3cm} | p{8cm}}
            sphere & \texttt{gv.addSphere('world/Sphere', radius, color=[r,g,b,a])} \\ \hline
            capsule & \texttt{gv.addCapsule('world/Capsule', radis, length, color)} \\ \hline
            box & \texttt{gv.addBox('world/Box', depth(x), length(y), height(z), color)}\\ \hline
            axis & \texttt{gv.addXYZaxis('world/Frame', color, radius, size)}
        \end{tabular} \newline

        \centerline{\textbf{Display}}
        \begin{tabular}{p{3cm} | p{8cm}}
            configuration & \texttt{gv.applyConfiguration('path/to/the/shape', [x,y,z,quaternion])} \\ \hline
            refresh & \texttt{gv.refresh()}
        \end{tabular} \newline

        \centerline{\textbf{Clean scene}}
        \begin{tabular}{p{3cm} | p{8cm}}
            erase world & \texttt{gv.deleteNode('world',True)}
        \end{tabular}

    \end{minipage}
};

\node[fancytitle, right=10pt] at (box.north west) {Gepetto Viewer};
\end{tikzpicture}
%------------ End Gepetto Viewer --------------

\end{multicols*}
\end{document}
